\documentclass[conference]{IEEEtran}
\usepackage[portuguese]{babel} % vai trocar automaticamente, por exemplo, Table por tabela
\usepackage{cite}
\usepackage{amsmath,amssymb,amsfonts}
\usepackage{algorithmic}
\usepackage{graphicx}
\usepackage{textcomp}
\usepackage{xcolor}
\usepackage{pgfplots}			%Importação de imagens tikz, carregar depois de xcolor
\usetikzlibrary{plotmarks}
%\usetikzlibrary{external}
%\tikzexternalize[prefix=tikz/]
\pgfplotsset{compat=newest}
\pgfkeys{/pgf/number format/.cd,1000 sep={}}%para não colocar , nos tikz para separar milhar

\usepackage[nolist]{acronym}
\usepackage{multirow}
\usepackage{booktabs}
\usepackage{verbatim}

\def\BibTeX{{\rm B\kern-.05em{\sc i\kern-.025em b}\kern-.08em
    T\kern-.1667em\lower.7ex\hbox{E}\kern-.125emX}}
\begin{document}

\begin{acronym}[TDMA]
    \acro{IEEE}{\textit{Institute of Electrical and Electronics Engineers}}
    \acro{ABNT}{Associação Brasileira de Normas Técnicas}
\end{acronym}

\newlength\figureheight
\newlength\figurewidth


\title{CI1064 Software Básico - Trabalho 1}

\author{\IEEEauthorblockN{Fabiano A. de Sá Filho}
\IEEEauthorblockA{\textit{Departamento de Informática} \\
\textit{Universidade Federal do Paraná -- UFPR}\\
GRR20223831 \\
fabiano.filho@ufpr.br}
}

\maketitle

\begin{abstract}
Este trabalho teve como objetivo a implementação de uma API proprietária de gerenciamento de alocação dinâmica de memória (HEAP). Esta API foi implementada em Assembly AMD-64, e suas funções podem ser acessadas através de um arquivo \textit{header}, incluso nos materiais deste trabalho. Neste artigo, discutirei brevemente alguns detalhes de implementação, bem como decisões de projeto relevantes.
\end{abstract}

\section{A API}
A API implementa e disponibiliza para o usuário as quatro funções a seguir:

\begin{verbatim}
void setup_brk(); // Obtém o endereço de brk

void dismiss_brk(); // Restaura o endereço de brk

void* memory_alloc(unsigned long int bytes);

// 1. Procura bloco livre com tamanho igual ou
 maior que a requisição
// 2. Se encontrar, marca ocupação, 
utiliza os bytes necessários do bloco, 
//    retornando o endereço correspondente
// 3. Se não encontrar, abre espaço para
um novo bloco

int memory_free(void *pointer); // Marca
 um bloco ocupado como livre
\end{verbatim}








\end{document}
